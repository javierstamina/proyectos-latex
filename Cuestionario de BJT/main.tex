\documentclass[12pt]{article}
\usepackage[utf8]{inputenc}
\usepackage{tikz}
\usepackage[american]{circuitikz}

\begin{document}
\pagestyle{empty}
\begin{center}
Unidad 2 de Electrónica Analógica\\
Cuestionario de BJT \\
\textbf{Valor de la evidencia de aprendizaje: 50 puntos}
\end{center} 
\vskip 1cm
Nombre completo: \rule{11cm}{0.2mm}
\vskip 1cm
\begin{enumerate}
\item Explique el funcionamiento y aplicaciones de los transistores de unión bipolar (\textit{BJT}).
\item ¿Qué tipo de transistores existen? Dibuje sus símbolos.
\item Deduzca la fórmula de \textit{$I_B$} para un circuito de polarización fija.
\item A partir de la configuración de polarización estabilizada en el emisor, deduzca la fórmula para calcular $V_{CE}$.
\item Explique el significado del parámetro $\beta$ de un transistor.
\item En el siguiente circuito calcule: a)$I_B$, b)$I_C$, c)$V_{CE}$, d)$V_C$.
\begin{figure}[h]
\begin{center}
\begin{circuitikz}
%\draw (0,0) grid (6,6);
\draw (3,5) to [short,-o] (3,6);
\node [above] at (3,6) {15 V};
\draw (1,5) -- (5,5);
\draw (1,5) to [R=390$k\Omega$] (1,2);
\draw (5,2) node[npn] (npn){};
\draw (5,5) to [R=3.2$k\Omega$] (5,2.75);
\draw (1,2) -- (4.3,2);
\draw (5,1.6) to node[ground]{} (5,0.7);
\end{circuitikz}
\label{fig:fija}
\end{center}
\end{figure}
\item Dibuje la curva característica del transistor \textit{BJT} para diferentes corrientes de base.
\item Explique los términos de voltaje y corriente de saturación.
\end{enumerate}


\end{document}
