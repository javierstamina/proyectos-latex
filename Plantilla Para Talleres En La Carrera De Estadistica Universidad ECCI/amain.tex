\documentclass[13 pts]{article}
\usepackage[utf8]{inputenc}
\usepackage[spanish]{babel}\spanishdecimal{.}
\usepackage{fancyhdr}
\usepackage{amsmath}
\usepackage{pifont}
\usepackage{vmargin}
\setpapersize{USlegal}
\usepackage{amssymb,latexsym}
\usepackage[mathscr]{euscript}
\usepackage{epsfig}
\usepackage{epic}
\usepackage{enumerate}
\usepackage{eepic}
\usepackage{amsmath}
\usepackage{amsthm}
\usepackage{threeparttable}
\usepackage{amscd}
\usepackage{here}
\usepackage{graphicx}
\usepackage{lscape}
\usepackage{tabularx}
\usepackage{array}
\usepackage{subfigure}
\usepackage{longtable}
\usepackage[usenames,dvipsnames]{pstricks}
\usepackage{epsfig}
\usepackage{listings}
\usepackage{pst-grad} % For gradients
\usepackage{pst-plot} % For axes
\usepackage{rotating}
\usepackage{bigints}
\usepackage{multicol} 
\usepackage{listings}
\usepackage{hyperref}
%Paquete para la escritura actuarial
\usepackage{actuarialsymbol}


\setlength{\textwidth}{16cm}
\setlength{\textheight}{18cm}
\usepackage[usenames,dvipsnames]{pstricks}
\topmargin = 0cm
\textheight = 11.4in
\textwidth = 6,8in
\oddsidemargin = 1.8cm
\evensidemargin = 0.0cm



\usepackage{fourier}
%\usepackage[left=2cm,right=2cm,top=2cm,bottom=2cm]{geometry}
\theoremstyle{definition}
\newtheorem{ex}{Ejercicio}
\usepackage{answers}
\Newassociation{sol}{Solution}{ans}
\renewcommand{\Solutionlabel}[1]{\textbf{solución #1.}}

\newtheorem{theorem}{Teorema}[section]
%\newtheorem{theorema}{Teorema}
\newtheorem{definition}[theorem]{ Definición}
\newtheorem{lemma}[theorem]{Lema}
\newtheorem{proposition}[theorem]{Proposición}
\newtheorem{corollary}[theorem]{Corolario}
\newtheorem{example}[theorem]{Ejemplo}
\newtheorem{exercise}[theorem]{Ejercicio}
%\newtheorem{defination}[theorem]{Definición}


\newcommand{\lf}[1]{$\displaystyle #1$}





\begin{document}




\voffset-2.5cm
\parbox{6cm}{\includegraphics [scale=0.8]{ecci.png} }\parbox{8cm}{{Universidad ECCI}\\{Programa de Estadística}\\{Curso que estudia}\\{Nombre de la actividad}\\{Nombre y apellidos} \bf  }

\vspace{-2.0cm}


\section*{En este espacio escribe la solución del taller}

\section*{Ecuaciones Matemáticas}
\label{sec:escuaciones}
Veremos una serie de tipos de expresiones matemáticas:\\

\(a^2 = b^2 + c^2\) o algo as? \(\bar{x^2} \ne \bar{x}^2\) \\

\begin{equation}
  \label{eq:suma}
  \sum_{0 \le i \ge  n} a_i = 1000
\end{equation}

A veces podemos colocar una ecuación en nuestro párrafo, como esta
 por ejemplo $\displaystyle x_{ij} = \sum_{0 \le i < n}\frac{x}{\sqrt[5]{1 + 23y^{112}}}$,o también puede estar al medio:
 
\[x_{ij} = \sum_{0 \le i < n}\frac{x}{\sqrt[5]{1 + 23y^{112}}}\]

También podemos definir integrales (en un cuadro con minipage): 

\fbox{
  \begin{minipage}{.9\textwidth} % ancho de la caja
    \begin{eqnarray*}
      \int{\frac{{Mx+N}}{{(x-a)^2+b^2}}dx}
    \end{eqnarray*}
  \end{minipage}
}\\[1em]


Y matrices:
\begin{equation}
  \label{eq:matrices}
  \left (
    \begin{array}[c]{ccc}
      a_{11} & a_{12} & a_{13} \\
      a_{21} & a_{22} & a_{23} \\
      a_{31} & a_{32} & a_{33} 
    \end{array}
  \right )^R
  =
  \left [
    \begin{array}[c]{cccc}
      b_{11} & b_{12} & \cdots & b_{1n} \\
      b_{21} & b_{12} & \cdots & b_{1n} \\
      \vdots & \vdots & \ddots & \vdots \\
      b_{n1} & b_{n2} & \cdots & b_{nn} 
    \end{array}
  \right ]
\end{equation}

Acotar:
\begin{equation}
  \label{eq:acotar}
  z =
  \left\{
    \begin{array}[c]{ll}
      0       & \mbox{si \(x < 0\)} \\
      2x + 1  & \mbox{si \(0 \le x \le 10\)} \\
      50x^5   & \mbox{si \(x > 0\)} \\
    \end{array}
  \right.
\end{equation}

Ojo con la numeración de las ecuaciones!!:
                                % {eqnarray*}  --> todos sin numeracion
\begin{eqnarray}
  H_n \approx \prod_{i=1}^n \frac{1}{\sqrt{2i}} = \frac{1}{\sqrt{2}}
  + \frac{1}{\sqrt{4}} +  \dots \nonumber \\
  H_n \approx \prod_{i=1}^n \frac{1}{\sqrt{2i}} =  \frac{1}{\sqrt{2}} + 
  \frac{1}{\sqrt{4}} +  \dots 
\end{eqnarray}

Algunos sombreros especiales:
\[\hat{x}, \check{x}, \breve{x}, \tilde{x}, \bar{x}, \vec{x}, \acute{x}, \grave{x}, \dot{x}, \ddot{x}\]


Símbolos importantes y ecuaciones varias:
%eqnarray: para ecuaciones en varias lineas
\begin{eqnarray*} 
  f: \mathbb{IR} \leftarrow \mathbb{C} \\
  f:A \times M \rightarrow M \\
  (\lambda, x) \longrightarrow \lambda x  \\
  \texttt{2}\>H_2\>+ O_2 \stackrel{\Delta}{\rightarrow} \texttt{2}\>H_2O\\
  \frac{\partial(X^2)}{\partial x} = 2x\\
  x^2 = y \longrightarrow x = \pm \sqrt{y}\\
  \overline{(x+y)} = (x+y)^c\\
  \Uparrow,\Downarrow, \Leftarrow, \Rightarrow\\
  t = r \cup s  \\
  t = \bigcup_{i=1}^n E[x]\\
\end{eqnarray*}

Esto es simpático:
\[ \underbrace{1 + \overbrace{2 + 3}^{5} + 4}_{10} \]

Esto más complejo:
\begin{displaymath}
  \int e^{x+y}dx dy \,\, = \,\,
  \int e^x e^y dxdy \,\, = \,\,
  \underbrace{\int e^x dx \, \int e^y dy}_{\stackrel{\stackrel{\uparrow}{Variables \; son \;independientes}}{Podemos \;realizar\; la\; integracion\; para\; x\; e\; y}}
\end{displaymath}



Ejemplo de bibliografía, use archivo .bib, para mayor información puede buscar en \url{http://logistica.fime.uanl.mx/miguel/docs/BibTeX.pdf}

\nocite{huertas2001calculo,franco2015calculo,giraldoactuaria}
\bibliography{biblio}{}
    \bibliographystyle{plain}









\end{document} 