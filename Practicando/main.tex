\documentclass{article}
\usepackage[utf8]{inputenc}
\usepackage{tikz}

\usepackage{amsmath, amsthm, amssymb}
\usepackage{tkz-euclide}
\usepackage{gensymb}
\usetikzlibrary{calc}

\usepackage{framed}
\usepackage{multicol}

\title{Practicando}
\author{javier argo}
\date{June 2022}
%margen de hoja
\usepackage[a4paper]{geometry}
\geometry{top=5cm, bottom=1.0cm, left=1.25cm, right=1.25cm}
%valro absoluto
\providecommand{\abs}[1]{\lvert#1\rvert}
\providecommand{\norm}[1]{\lVert#1\rVert}
%secante hiperbolico
%\providecommand{\sech}[1]{1/\cosh}
\usepackage{pst-all}
%\usepackage{pgfplots}
\usepackage{infix-RPN,pst-func}
\usepackage{pst-infixplot}
\usepackage{pst-node}
\usepackage{pst-plot}
%\usepackage[active,tightpage]{preview}
%\PreviewEnvironment{pspicture}
%\PreviewBorder=10pt\relax


\begin{document}
\tableofcontents
\listoffigures
\maketitle
\section{Integrales de funciones hiperbólicas}
\begin{multicols}{2}
\fbox{\begin{minipage}{15em}
\underline{NOTA:}
$$\sinh(x)=\frac{\epsilon^x - \epsilon ^{-x}}{2}$$
$$\cosh(x)=\frac{\epsilon^x + \epsilon ^{-x}}{2}$$
\end{minipage}}
\\ \\
\[ \int \sinh(x) dx = \cosh(x) + C \]
\[ \int \cosh(x) dx = \sinh(x) + C \]
\[ \int \tanh(x) dx = \ln \abs{\cosh(x)} + C \]
\[ \int \coth(x) dx = \ln \abs{\sinh(x)} + C \]
\[ \int \frac{1}{\cosh(x)} dx = \arctan\sinh(x) + C \]
\[ \int \frac{1}{\sinh(x)} dx = \ln \tanh(\frac{x}{2}) + C \]
\[ \int \frac{1}{\cosh^{2}(x)} dx = \tanh(x) + C \]
\[ \int \frac{1}{\sinh^{2}(x)} dx = -\coth(x) + C \]
\end{multicols}

\section{Integrales de funciones logarítmicas}
\begin{multicols}{2}
\begin{equation*}
\begin{aligned}
\int \ln(x) dx = x \ln(x) - x + C
\end{aligned}
\end{equation*}

\begin{equation*}
\begin{aligned}
\int (\ln(x))^{2} dx = x (\ln(x))^{2}-2x\ln(x)\\ +2 x + C \\
\end{aligned}
\end{equation*}

\begin{equation*}
\begin{aligned}
\int (\ln(x))^{3} dx = x (\ln(x))^{3}- 3x (\ln(x))^{2}\\+6x\ln(x) -6x + C
\end{aligned}
\end{equation*}

\begin{equation*}
\begin{aligned}
\int (\ln(x))^{n} dx = x (\ln(x))^{n}\\- x \int(\ln(x))^{n-1} dx + C
\end{aligned}
\end{equation*}

\begin{equation*}
\begin{aligned}
\int\frac{1}{\ln(x)}  dx =\ln\ln(x)+\ln(x)\\+\frac{(\ln(x))^{2}}{2\cdot2!}+\frac{(\ln(x))^{3}}{3\cdot3!} +\hdots+ C
\end{aligned}
\end{equation*}

\begin{equation*}
\begin{aligned}
\int\frac{1}{(\ln(x))^{n}}  dx = -\frac{x}{(n-1)(\ln(x))^{n-1}} \\+\frac{1}{n-1} \int\frac{1}{(\ln(x))^{n-1}}  dx+C
\end{aligned}
\end{equation*}

\end{multicols}
\newpage

\section{Funciones trigonométricas}
\subsection{Función Seno}
Tiene por regla de correspondencia:
$y = \sin x$\\
\underline{Gráfica}\\

\begin{center}
    

\begin{pspicture}(-0.5,-1.25)(3.5,1.25)
\infixtoRPN{sin(x*180/3.141592654)}
% Fijarse en como cambiamos la
% variable x a radianes!!!
\psplot[linecolor=red,linewidth=1.5pt]{0}{6.283185308}{\RPN}
\psaxes[xunit=3.141592654,trigLabels=true]{->}(0,0)(-0.5,-1.5)(2.1,1.5)
\end{pspicture}
\end{center}
    
\begin{center}
\underline{Dominio} ($D_f$)\\
$D_f = R$\\
\underline{Rango} ($R_f$)\\
$R_f = [1:1]$\\
\underline{Periodo} ($T$)\\
$T = 2\pi$

\end{center}

\end{document}
