%%%%%%%%%%%%%%%%%%%%%%%%%%%%%%%%

% Introduction 

%%%%%%%%%%%%%%%%%%%%%%%%%%%%%%%%

\chapter[Introduction]{\textbf{I}ntroduction}\label{sec:intro}
\sectionblue*{Summary}
\begin{center}
\begin{tabular}{c}
\fcolorbox{blue}{verylightgrey}{
\begin{minipage}[][4cm][c]{0.8\linewidth}
\sffamily
%abstract 
This chapter details the general questions of the document. It provides a brief 
explanation of the motivation for this work. Then, the list of objectives of the
thesis is introduced. The completion of these tasks and the final calendar
of execution of the project (year by year) is included as well. The manuscript
is logically divided into three different parts: Preliminaries, State of the Art
and Meta-alignments. There is also a brief description of the chapters of each
part. Finally, some particular considerations about how to read the book and
the layout of the document are presented.
\end{minipage}}\\
\\[2ex]
\begin{minipage}[][4cm][c]{0.9\linewidth}
\minitoc
\end{minipage}
\end{tabular}
\end{center}
\newpage

\sectionblue{General objectives}
\index{thesis!genobjectives@general objectives}%

\lettrine[lines=4,loversize=-0.1,lraise=0.1,lhang=.2]{T}{he principal objective of the thesis dissertation}
is to explain in detail the topic on which the work of several years has been focused. In addition, 
the experience of the author at different areas has been reflected here in the numerous descriptions 
and solutions to several biological and computational problems. Speculation about future research and
criticism have been a valuable ingredient as well.

This is a thesis about computational sequence analysis, particularly applied to characterize genomic
sequences. The way in which this synergy between a biological problem and a computational solution 
is expressed was considered to be crucial for the success of this document. The generality of the 
proposed solutions, which can be applied to any type of sequence (biological or not), is also underlined
in the corresponding sections.

The core of the thesis is the development of a new family of algorithms to align transcription regulatory 
regions. Among them, a global pairwise algorithm and a global progressive multiple algorithm have been 
shown to be useful in the characterization of a gene promoter region, specially when the amount of 
predictions by other systems is excessive. Sketches of other versions are also provided (parallel, local).

The work performed about the meta-alignment strategy has been interestingly complemented and enriched
with a serious approach to the algorithms that originated the concept of sequence analysis several decades
ago. Such a chapter is an interesting opportunity to review for the first time some of the classic papers 
in the field that are still very relevant, in spite of the deluge of new proposals and publications 
continuously released. The introduction of this material in the document improves without any doubt the 
quality of the final manuscript.

In addition, several references about the relationship between current advances in genomics and society 
can be found in the text. In my opinion, ethics must be part of any human achievement. Genomics and other 
'Omics' disciplines promise to radically change our way of life. Medicine, biotech farming, crime 
investigation and personal privacy among others will be severely affected.

To sum up, this thesis aims to become an educational book reference. This is an excellent opportunity to 
explain in detail the topic of the meta-alignment but also to construct an exciting portrait of sequence 
analysis in computational biology. To satisfy all of these requirements, the use of current technologies 
to produce an outstanding work was also mandatory. Thus, a DVD with additional materials (electronic 
thesis, relevant bibliography, source code, educational material, \ldots) supporting the main text is a
good complement to the PhD dissertation.


\sectionblue{Objectives}
\index{thesis!objectives@objectives}% 

The characterization of gene regulatory regions is a fundamental step toward understanding the
great existing variability between different species. However, it is still an open problem
due to the peculiar features of the regulatory elements. The research in this PhD thesis has
been oriented to the development of new computational methods of alignment to deal with such 
information. However, it is important to mention that the algorithms presented here can deal 
with other problems that show a similar theoretical framework, lacking of a biological background.

\noindent In short, the following objectives were established in 2001 for this thesis:

\begin{menumerate}
\item
To study the biological problem of gene regulation in eukaryotes. This includes the control of gene 
expression, specially through the transcription of the genes: promoters, transcription factors, 
DNA-protein binding sites, chromatin effect, CpG islands.
\item
To analyze the current computational methods to search regulatory elements in a promoter region. This 
includes the algorithms based on pattern matching using catalogues of regulatory elements and the 
pattern discovery algorithms that extract useful information from a set of related sequences.
\item
To investigate the more recent comparative approaches based on phylogenetic footprinting and 
microarrays. To understand the biological concepts behind the gene orthology. To study the biological 
and technological concepts of the high-throughput expression experiments.
\item
To analyze the existent sequence pairwise sequence alignment algorithms. To study the concept
of map, the mapping functions and the map alignment problem.
\item
To design novel algorithms to align two regulatory sequences that produce a minor amount of false 
positives. To present real biological scenarios in which these approaches show to be more efficient 
than the conventional sequence alignment algorithms.
\item
To compile and to maintain a public dataset of regulatory annotations suitable for training these
and other algorithms that deal with data from comparative genomics and microarray experiments.
\item
To study several alternatives to extend the basic pairwise approach developed before to align
multiple sequences. Test this approach on orthologous datasets and microarray expression data.
\item
Public distribution of the software and the databases produced during this thesis to the 
scientific community. To write web servers that implement most of the methods presented above.
\end{menumerate}


\sectionblue{Thesis chronology}
\index{thesis!chronology@chronology}% 

This is a short enumeration of the main tasks implemented during the PhD thesis and their 
associated results, year by year:

\begin{mitemize}
\item 2001
\begin{menumerate}
\item Planning: decide the main lines and the objectives of the thesis.
\item Biological problem: bibliographical research in general molecular biology books about the 
eukaryotic transcription and other forms of gene regulation.
\item State of the art: bibliographical research in published papers about the classical algorithms
and strategies to analyze gene promoter regions. Including the study of the advanced techniques such
phylogenetic footprinting and microarray experiments.
\item Attended conferences: Intelligent Systems in Molecular Biology (ISMB) at Copenhaguen, Denmark.
\end{menumerate}

\item 2002
\begin{menumerate}
\item Analysis of co-expressed genes in \emph{Drosophila melanogaster}: gene characterization, 
G+C content, clustering, gene function, promoter characterization including phylogenetic analysis.
\item Analysis of co-expressed genes in \emph{Mus musculus}: the results of several microarrays were 
analyzed with the existing computational tools, including phylogenetic footprinting.
\end{menumerate}

\item 2003
\begin{menumerate}
\item
Developing the global and local meta-alignment first prototypes.
\item
Bibliographical research to find regulatory data for training the meta-alignment approach.
\item
Attended conferences: Research in Computational Biology (RECOMB) at Berlin, Germany.
\end{menumerate}


\item 2004
\begin{menumerate}
\item
Tuning the meta-alignment. Improving the efficiency of the basic implementation with lists.
\item
Writing the web server of the pairwise meta-alignment program.
\item 
Training the meta-alignment on a small dataset of annotated promoters.
\item
First prototypes for multiple meta-alignment.
\item
Attended conferences: Systems Biology at Cold Spring Harbor Labs, New York, USA.
\end{menumerate}


\item 2005
\begin{menumerate}
\item
Creation of a public database of annotated promoters (ABS).
\item
Final tests: pairwise meta-alignment approach on the CISRED database.
\item
Evaluation of the quality of weight matrices using the meta-alignment.
\item
Tuning the multiple meta-alignment. Improving the computational efficiency.
\item
Variations to allow the existence of non-colinear alignments in the results.
\item
Starting to write the thesis dissertation.
\item
Attended conferences: Systems Biology at Cold Spring Harbor Labs, New York, USA.
\end{menumerate}

\item 2006
\begin{menumerate}
\item
Final training of the multiple meta-alignment on a set of orthologous of multiple species.
\item
Finishing the thesis dissertation.
\item
Public defense of the PhD thesis.
\end{menumerate}
\end{mitemize}



\sectionblue{Outline of this thesis}
\index{thesis!outline@outline}% 

This thesis has been written following the format of a text book. The main text is divided into
three parts: introduction, state of the art and results. Every part consists of a set of 
chapters, each one devoted to a given topic. Chapters can be read separately to facilitate the accession 
to individual parts of the book, but the thesis has been written following a linear and continuous 
logical script.

\noindent This is a brief description of the content of each chapter:

\begin{menumerate}
\item Introduction: general motivation of the thesis containing the objectives, the calendar
and other considerations about the project and the format of the book.

\item The post-genomic era: biological description of genomic concepts (genes, DNA, mRNA), 
the genome sequencing projects, bioinformatics, future implications of the genomic research in medicine.

\item The golden age of sequence analysis: a comprehensive historical review of the pioneering algorithms 
in sequence and map alignment in the seventies and eighties, including a detailed analysis of the most 
relevant ones.

\item Computational gene and promoter characterization: a survey of the state of the art in the
analysis of genomic sequences (genes and regulatory regions), and a study of the different techniques
implemented such as the representation of signals, the detection of biased content regions or the
homology search.

\item Pairwise meta-alignment of regulatory sequences: the mapping functions, the TF-map approach,
basic implementations, the accurate construction of collections of examples, the training, the application
on a database of co-regulated genes, the detection of promoter regions, the use of meta-alignment to 
evaluate the specificity of matrices. Other versions: local and parallel meta-alignment.

\item Multiple meta-alignment of regulatory sequences: the progressive approach, the design of the final 
solution, the modification to produce non-colinear alignments, the tests on orthologous promoters from 
multiple species.

\item Conclusions: the enumeration of the results of this thesis.

\item Appendix section: curriculum vitae, software and web servers, publications, posters, web glossary.

\end{menumerate}


\sectionblue{Particular considerations}

The following are some individual considerations about the thesis:

\begin{mitemize}
\item
The electronic version of this document has hyper links for the table
of contents, for the bibliographic references, but most important of all, 
also for the web addresses on the Internet---from now on, their Uniform Resource 
Locator (URL).
This means that you can visit the corresponding web page by clicking
your pointer on them, in case that you have your PDF viewer properly
customized. Many of the URLs presented in this book have been
collected in a web links reference index available on
page~\pageref{sec:webrefs}. URLs within paragraphs have been moved
into that web glossary in order to avoid unbalanced line breaks and
for a more pleasant reading. A reference to the corresponding page in
the web reference index is provided instead. 
\item
An attempt has been made to keep software names as provided by their
authors. Those names appear in a \prog{\progfont}. Database names are
typeset in a \db{\dbfont}. A \emph{emphasized font} was used for gene
names.
\item
The first time an acronym appears in the document, the full name will
be provided and the acronym itself will be shown in parentheses. 
\item
The publications and submissions of papers in which the author of this thesis 
was involved are included at the end of the thesis as an appendix.
\item
The use of colour is considered to be essential to accurately highlight
some contents of the thesis such as the equations, the algorithms or the
figures and the tables. 
\item
The author of this thesis has carefully selected the bibliography of each
chapter. Following such references, a detailed reconstruction of such a
topic can be performed with great accuracy. Some of these references are
also included as electronic supplementary material in the DVD companion 
to this thesis.
\end{mitemize}

%%% References for this chapter
%%% ENCERRAR ENTRE LLAVES PARA EVITAR PROBLEMAS
%\bibliographystyle{plainnat}
%{\bibliography{sections/bibliography}}
