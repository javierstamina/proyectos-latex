\chapter[Conclusions]{\textbf{C}onclusions}\label{sec:conclusions}

\index{thesis!concl@conclusions}
\lettrine[lines=4,loversize=-0.1,lraise=0.1,lhang=.2]{T}{he TF-map alignments can be very useful} to
efficiently perform searches of promoter elements that might be conserved in different species. In short, 
the research presented here has contributed to improve the computational characterization of gene 
transcription regulatory regions in the following aspects:

\begin{menumerate}
\item
We have designed a new family of algorithms, which are named TF-map alignments or simply 
meta-alignments, to detect conserved high-order configurations of functional elements that do not show 
discernible sequence conservation. The meta-alignment algorithm does not directly 
compare the primary sequences. Instead, the algorithm aligns the map of high-level
elements obtained with an external mapping function over the original sequences, taking 
into account their position, the element class and the mapping score.

\item
We have generalized the pairwise meta-alignment algorithm to deal with multiple maps.
We followed a progressive approach in which the multiple meta-alignment is build up in
a stepwise manner: a first multiple alignment is created with the two most similar maps,
and the rest of maps or groups of maps are then aligned to this initial multiple meta-alignment
following a guide tree.

\item
We have investigated the structure and the shape of the resulting meta-alignments. We have
incorporated some modifications in the basic algorithm in order to detect non-collinear 
configurations in the alignments without additional computational cost.

\item
We have successfully applied the meta-alignment algorithms on the biological problem of 
eukaryotical promoter characterization. First, we have manually curated a collection of
orthologous transcription factor binding sites from the literature, that are experimentally 
verified in human, mouse, rat or chicken. Next, we have trained the meta-alignment program
on a subset of well characterized human-mouse promoters, extracted from this collection.
Then, we have shown the TF-map alignments are more accurate than conventional sequence 
alignment to distinguish pairwise gene co-expression in a large collection of microarray results. 

\item
We have also used the meta-alignment approach to distinguish promoters from other 
gene regions in a set of well characterized human-rodent gene pairs and their corresponding 
orthologs in chicken and zebrafish. In this particular problem, the multiple meta-alignment 
identified correctly most orthologous promoter regions, even when comparing to protein coding
regions that presented a stronger sequence conservation.

\item
We have comprehensively reviewed the topic of sequence alignment, specially focusing on the 
pioneering algorithms that have mostly contributed to the field. In addition, we have also 
contributed to extend our expertise in the areas of computational gene finding and promoter 
characterization, within the field of bioinformatics.
\end{menumerate}

