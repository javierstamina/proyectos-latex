%%
%% $Id: miscellanea.tex 11 2005-03-29 00:58:58Z jabril $
%%

\noappendix{Miscellanea}

This thesis layout is largely derived from the \LaTeX\ template
created by Robert Castelo in 2002\footnote{R. Castelo, April
2002.\newline\hspace*{1cm}''The Discrete Acyclic Digraph Markov Model in Data
Mining''\newline\hspace*{1cm}Faculteit Wiskunde en Informatica, Universiteit
Utrecht}.  His templates were extended by Sergi Castellano and
Gen\'{\i}s Parra for their theses. Josep Francesc Abril substantially 
improved those files, creating an excellent automatical framework that 
produces a variety of different formats and layouts. Here, I provide some 
comments on his version and the modifications I incorporated to, and the 
source code for download.

\index{software!typesetting!latex@\LaTeX|(}%
\sectionred*{Technical comments}

This book was typeset with GNU \prog{emacs} 21.3.1 in \LaTeX\ mode and
converted to PDF with \prog{pdflatex} 3.14159-1.10b (Web2C 7.4.5). All
running on a linux box with Red Hat Fedora Core 2 and kernel
2.6.9-1.6. \LaTeX\ is a document preparation system, powerful, robust
and able to achieve professional results \citep{lamport:1994a}. However,
the learning curve may be stiff. 

The main document, \prog{thesis.tex}, depends on several \LaTeX\
files---including each chapter, the tables and few \ps{} figures---,
but it also depends on other files---such as style files, hacked
\LaTeX\ packages, several bitmaps and the PDF files for the attached
papers. Furthermore, \prog{pdflatex} had to be run several times,
\index{software!typesetting!pdflatex@\prog{pdflatex}}%
together with \BiBTeX\ (to produce the bibliography chapter),
\index{software!typesetting!bibtex@\BiBTeX}%
\prog{makeindex} (to build the index and the web glossary), 
\prog{thumbpdf} (to generate the main PDF document thumbnails),
\index{software!typesetting!thumbpdf@\prog{thumbpdf}}%
and few \perl{} scripts. A \prog{Makefile} was written to automatize
the compilation process of the whole document. In fact, the
\prog{Makefile} was extended to produce four versions of the main
document. The ``\textsl{draft}'' version does not include figures and
the PDF files for the papers, displaying crop marks and boxes
around several elements (such as the area reserved for the
pictures). The ``\textsl{proofs}'', where everything is included but
crop marks and boxes are kept, and different hyperlink types use
different colors. The ``\textsl{pdf}'' version is the electronic
version in which all the hyperlinks are marked in blue color, crop
marks are disabled. Finally, the ``\textsl{press}'' version is very
similar to the ``\textsl{pdf}'' one, currently the only difference
is that all the hyperlinks are black. The \prog{Makefile}
also includes a rule to build the final book ``\textsl{cover}'', which
recycles the \prog{abstract.tex} file and takes some customization
from the same style file as the main \prog{thesis.tex} file.

The compilation of a complete version of this document takes about
\units{600}{seconds}---of course, the ``\textsl{draft}'' version takes much
less---with an AMD Athlon 64 processor 3200+, with \units{512}{KB} of
RAM. This is mainly due to the several steps required to ensure that
every reference, index and so on, is in place. The basic build series
of commands is the following: an initial \prog{pdflatex}, a \BiBTeX\
run to produce the bibliography, a second run of \prog{pdflatex} to
include it, one call to \prog{makeindex} (for the Web Glossary), 
a third run of \prog{pdflatex} to include the
glossary, another call to \prog{makeindex} (to generate the final
index) and to \prog{pdflatex}, then \prog{makeindex} and
\prog{pdflatex} are run again, an extra run of \prog{pdflatex} is
followed by \prog{thumbpdf}, and a final \prog{pdflatex} to obtain the
finished document. If any problem was found, like missing references,
an extra round of \prog{pdflatex}, \BiBTeX\ and \prog{pdflatex} is
performed by the \prog{Makefile}.

Here you can find the version of some of the programs refereed above:
\BiBTeX\ version 0.99c (Web2C 7.4.5), \prog{thumbpdf} version 3.2
(2002/05/26), and \prog{makeindex} version 2.14 (2002/10/02).


\sectionred*{\LaTeX\ Packages}

As there are four versions of the document, the \prog{ifthen} package was
used to define version specific parameters, as well as to include
different files. The package \prog{geometry} facilitates the
definition of the page layout. The current document original dimensions
for both, the electronic and printed versions, are \units{170}{mm}
width by \units{240}{mm} height.  The ``\textsl{cover}'' requires
\prog{calc} to calculate automatically the total width for the page
layout, which includes the front and the back covers and the spine
width. The main document basic font size is the default value for the
``\texttt{book}'' document class, \units{10}{pt}.

The \prog{crop} package is usefull to define the trimming marks for
the ``\textsl{draft}'' and ``\textsl{proofs}'' versions of this
document. It distinguishes between the logical page, the page sizes
defined by the user, and the physical page, the page size for the
hardcopy. The \prog{layout} package is used in the ``\textsl{draft}''
version to show on the first page the \LaTeX\ variable settings
controlling the page layout. Another useful package has been
\prog{nextpage}, which provides additional
``\prog{clear{\ldots}page}'' commands that ensure to get empty even
pages at the end of chapters---and of course, to ensure that all
chapters begin at odd pages---, even with automatically generated
sections like the Bibliography and the Index.

The \prog{babel} package provides a set of options that allow the user
to choose the language(s) in which the document will be typeset, for
instance language-specific hyphenation patterns. The default language
was set to ``\texttt{english}'', while ``\texttt{catalan}'' and
``\texttt{spanish}'' were also loaded for using them for the
corresponding translations of the \textsc{Abstract}.

When working with \prog{pdflatex} there are three unvaluable packages:
\prog{pdfpages}, which makes it easy to embed external PDF documents,
such as the attached publications;
\prog{thumbpdf}, it must be included in files for which a user wants
to generate thumbnails (which are created by the \prog{thumbpdf}
program); and \prog{hyperref}, which extends the functionality of all
the \LaTeX{} cross-referencing commands to produce \prog{special}
commands which a driver can turn into hypertext links.  To protect URL
characters we must load the \prog{url} package, unless we have already provided
\prog{hyperref}. This package has its own version of the \prog{url}
macro, enhanced to provide clickable URLs.

To include \ps{} figures one needs \prog{graphics} and/or
\prog{graphicx}. Those packages are modified by
\prog{pdflatex} so that they are able to include bitmaps
(PNGs, JPEGs, and so on) and PDF files into the document. \prog{color}
facilitates the specification of user-defined colors (such as the
cover green shades). Figures generated with \LaTeX\ can use any of the
following packages: \prog{pstricks}, \prog{pstcol}, \prog{multido}.

The bibliography was produced with \BiBTeX. The package \prog{natbib}
(NATural sciences BIBliography) provides both author-year and
numerical citations; it makes possible to define different
citation styles. We have set the following options:
``\texttt{round}'', to put citations within parenthesis;
``\texttt{colon}'', to separate multiple citations with colons;
``\texttt{authoryear}'' to show author and year citations (instead of
numerical citations); and the option ``\texttt{sectionbib}'' to use
the package \prog{chapterbib}. The style ``\texttt{plainnat}'' was then applied
to format the bibliography. The package \prog{chapterbib} allows to include
a bibliography for each chapter. The package \prog{minitoc} creates
a mini table of contents for each chapter as well. 

\prog{makeidx} provides the macros required to make a subject
index. To show the capital letter section headings, few variables were
redefined on an auxiliary file (\prog{header.ist}).
One glossary was generated for this document: the web references. 
The package \prog{glossary} allowed us to customize the format of this 
section.

We also defined a style file named \prog{mythesis.sty}. It loads the
following font packages: \prog{fontenc} (with ``\texttt{T1}'' option),
to set extended font encoding (accents and so on); \prog{textcomp}, to
include some extra symbols, such as the Euro symbol for instance;
\prog{pifont}, for \textsc{Symbol} and \textsc{Zapf Dingbats} fonts;
\prog{charter}, with which roman family is set to \textsc{BitStream-Charter}; 
\prog{helvet}, with which sans-serif family is set to \textsc{Helvetica}; 
\prog{euler}, with which formulas are set to \textsc{Euler}; 
and \prog{courier}, to set typewriter family to \textsc{Courier}. 
Other packages that were loaded are: \prog{fancyhdr}, to produce nice
headings; \prog{fancyvrb}, to extend the \prog{verbatim} environment;
\prog{comment}, to hide parts of the original \LaTeX\ files;
\prog{rotating}, to rotate boxes of text; and \prog{multirow}, to get
multirow cells within the \prog{tabular} environment.
%; and \prog{multicol} 
\index{software!typesetting!latex@\LaTeX|)}%


\sectionred*{Getting the template files}

You are free to copy, modify and distribute the template files of this
thesis, under the terms of the GNU Free Documentation License as
published by the Free Software Foundation.  Any script bundled in this
distribution, including the \prog{Makefile}, is under the terms of the
GNU General Public License.\index{GNU-GPL} The template for this thesis
as well as the DVD related files are available from:

\centerline{\url{http://genome.imim.es/~eblanco/MyThesis/}}


%%%%%%%%%%%%%%%%%%
%%% References for this chapter
%%% ENCERRAR ENTRE LLAVES PARA EVITAR PROBLEMAS
\bibliographystyle{plainnat}
{\bibliography{sections/bibliography}}

