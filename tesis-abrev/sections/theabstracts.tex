\chapter*{\textcolor{blue}{\textbf{A}bstract}}
\addstarredchapter{Abstract}

\vskip -2ex
% English version
\begin{small}
%BACKGROUND
The sequences are very versatile data structures. In a straightforward manner,
a sequence of symbols can store any type of information. Systematic analysis
of sequences is a very rich area of algorithmics, with lots of successful
applications. The comparison by sequence alignment is a very powerful analysis 
tool. Dynamic programming is one of the most popular and efficient approaches 
to align two sequences. However, despite their utility, alignments
are not always the best option for characterizing the function of two
sequences. Sequences often encode information in different levels of
organization (meta-information). In these cases, direct sequence comparison
is not able to unveil those higher-order structures that can actually explain
the relationship between the sequences.

%METHODS
We have contributed with the work presented here to improve the way in which
two sequences can be compared, developing a new family of algorithms that
align high level information encoded in biological sequences (meta-alignment).
Initially, we have redesigned an existent algorithm, based in dynamic programming,
to align two sequences of meta-information, introducing later several improvements
for a better performance. Next, we have developed a multiple meta-alignment algorithm,
by combining the general algorithm with the progressive schema. In addition,
we have studied the properties of the resulting meta-alignments, modifying the
algorithm to identify non-collinear or permuted configurations.

%RESULTS
Molecular life is a great example of the sequence versatility. Comparative genomics
provide the identification of numerous biologically functional elements. The
nucleotide sequence of many genes, for example, is relatively well conserved
between different species. In contrast, the sequences that regulate the gene
expression are shorter and weaker. Thus, the simultaneous activation of a set
of genes only can be explained in terms of conservation between configurations
of higher-order regulatory elements, that can not be detected at the sequence level.
We, therefore, have trained our meta-alignment programs in several datasets
of regulatory regions collected from the literature. Then, we have tested the
accuracy of our approximation to successfully characterize the promoter regions
of human genes and their orthologs in other species.
\end{small}

\clearemptydoublepage

\chapter*{\textcolor{blue}{\textbf{R}esumen}}
\addstarredchapter{Resumen}
\begin{small}
%BACKGROUND
Las secuencias son una de las estructuras de datos más versátiles que
existen. De forma relativamente sencilla, en una secuencia de símbolos
se puede almacenar información de cualquier tipo. El análisis sistemático
de secuencias es un área muy rica de la algorítmica, con numerosas aproximaciones
llevadas a cabo con éxito. En concreto, la comparación de secuencias mediante
el alineamiento de éstas es una herramienta muy potente. Una de las aproximaciones
más populares y eficientes para alinear dos secuencias es el uso de la programación
dinámica. Sin embargo, a pesar de su evidente utilidad, un alineamiento de dos 
secuencias no es siempre la mejor opción para caracterizar su función. Muchas veces,
las secuencias codifican la información en diferentes niveles (meta-información).
Es entonces cuando la comparación directa entre dos secuencias no es capaz de
revelar aquellas estructuras de orden superior que podrían explicar la relación
establecida entre éstas.

%METHODS
Con este trabajo hemos contribuído a mejorar el modo en el que dos secuencias
pueden ser comparadas, desarrollando una familia de algoritmos de alineamiento
de la información de alto nivel codificada en secuencias biológicas
(meta-alineamientos).  Inicialmente, hemos rediseñado un antiguo algoritmo, basado en 
programación dinámica, capaz de alinear dos secuencias de meta-información, 
procediendo despues a introducir varias mejoras para acelerar su velocidad.
A continuación hemos desarrollado un algoritmo de meta-aliniamento capaz de alinear 
un número múltiple de secuencias, combinando el algoritmo general con un esquema de 
clustering jerárquico. Además, hemos estudiado las propiedades de los meta-alineamentos 
producidos, modificando el algoritmo para identificar alineamientos con una 
configuración no necesariamente colineal, lo que permite entonces la detección
de permutaciones en los resultados.

%RESULTS
La vida molecular es un ejemplo paradigmático de la versatilidad de las secuencias. 
Las comparaciones entre genomas, ahora que su secuencia está disponible,
permiten identificar numerosos elementos biológicamente funcionales. La secuencia 
de nucleótidos de muchos genes, por ejemplo, se encuentra aceptablemente conservada 
entre diferentes especies. En cambio, las secuencias que regulan la expresión de los 
propios genes son más cortas y variables. Así que la activación simultanea de un 
conjunto de genes se puede explicar sólo a partir de la conservación de configuraciones 
comunes de elementos reguladores de alto nivel, y no a partir de la simple conservación
de sus secuencias. Por tanto, hemos entrenado nuestros programas de 
meta-alineamiento en una serie de conjuntos de regiones reguladoras recopiladas
por nosotros mismos de la literatura y despues, hemos probado la utilidad
biológica de nuestra aproximación, caracterizando automáticamente con éxito
las regiones activadoras de genes humanos conservados en otras especies.
\end{small}

\clearemptydoublepage

\chapter*{\textcolor{blue}{\textbf{R}esum}}
\addstarredchapter{Resum}

\begin{small}
%BACKGROUND
Les seqüències són una de les estructures de dades més versàtils que existeixen. De 
forma relativament senzilla, en una seqüència de símbols es pot emmagatzemar informació 
de qualsevol tipus. L' anàlisi sistemàtic de seqüències es un àrea molt rica de 
l'algorísmica amb numeroses aproximacions desenvolupades amb éxit. Particularment, la 
comparació de seqüències mitjançant l'alineament d'aquestes és una de les eines més 
potents. Una de les aproximacions més populars i eficients per alinear dues seqüències 
es l'ús de la programació dinàmica. Malgrat la seva evident utilitat, un alineament de 
dues seqüències no és sempre la millor opció per a caracteritzar la seva funció. Moltes 
vegades, les seqüències codifiquen la informació en diferents nivells (meta-informació). 
És llavors quan la comparació directa entre dues seqüències no es capaç de revelar aquelles 
estructures d'ordre superior que podrien explicar la relació establerta entre aquestes 
seqüències. 

%METHODS
Amb aquest treball hem contribuït a millorar la forma en que dues seqüències
poden ser comparades, desenvolupant una família d'algorismes d'alineament de la 
informació d'alt nivell codificada en seqüències biològiques 
(meta-alineaments).  Inicialment, hem redissenyat un antic algorisme, basat en 
programació dinàmica, que és capaç d'alinear dues seqüencies de meta-informació, 
procedint després a introduir-hi vàries millores per accelerar la seva velocitat. 
A continuació hem desenvolupat un algorisme de meta-aliniament capaç d'alinear 
un número múltiple de seqüències, combinant l'algorisme general amb un esquema de 
clustering jeràrquic. A més, hem estudiat les propietats dels meta-alineaments 
produïts, modificant l'algorisme per tal d'identificar alineaments amb una 
configuració no necessàriament col.lineal, el que permet llavors la detecció 
de permutacions en els resultats.

%RESULTS
La vida mol.lecular és un exemple paradigmàtic de la versatilitat de les seqüències. 
Les comparacions de genomes, ara que la seva seqüència està disponible,
permeten identificar numerosos elements biològicament funcionals. La seqüència 
de nucleòtids de molts gens, per exemple, es troba acceptablement conservada 
entre diferents espècies. En canvi, les seqüències que regulen l'expressió dels 
propis gens son més curtes i variables. Així l'activació simultànea d'un conjunt
de gens es pot explicar només a partir de la conservació de configuracions comunes
d'elements reguladors d'alt nivell, i no pas a partir de la simple conservació
de les seves seqüències. Per tant, hem entrenat els nostres programes de 
meta-alineament en una sèrie de conjunts de regions reguladores recopilades
per nosaltres mateixos de la literatura i desprès, hem provat la utilitat 
biològica de la nostra aproximació, caracteritzant automàticament de forma exitosa 
les regions activadores de gens humans conservats en altres espècies.
\end{small}

%\clearemptydoublepage

%\nochapter{Resumen}
%\vskip -2ex
%\begin{small}
%  \input{sections/resumen} % Spanish version
%\end{small}
