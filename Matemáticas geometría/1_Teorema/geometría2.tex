%*********************
% EJERCICIO

\section{Teoremas de triángulos}
\subsection{Teorema incentro baricentro}

\begin{figure}[h]
\centering 
\begin{tikzpicture}[scale=0.75]
\tkzDefPoints{-3/0/A, -3.5/4/B, 3/0/C}
\tkzDrawPolygon(A,B,C)
\tkzDrawPoints[color=blue](A,B,C)
\tkzLabelPoints(A,C)
\tkzLabelPoints[above](B)
\tkzInCenter(A,B,C) \tkzGetPoint{I}
\tkzCentroid(A,B,C) \tkzGetPoint{G}
\tkzDrawPoints[color=blue](I,G)
\tkzLabelPoints[below](I,G)
\tkzDrawSegments(B,I B,G I,G)
\tkzMarkAngle[size=0.7](A,B,I)
\tkzLabelAngle(A,B,I){$\alpha$}
\tkzMarkAngle[size=0.7](I,B,C)
\tkzLabelAngle(I,B,C){$\alpha$}
\end{tikzpicture}

\caption{}
   
\end{figure}
Sea el triángulo $\triangle ABC$, de incentro $I$ y baricentro $G$.\\
Si $\overline{IG} \parallel  \overline{AC}$. Entonces:
%formulas
$$ AC = \frac{AB+BC}{2} $$


%*********************
% EJERCICIO
\subsection{Teorema de área y bisectriz interior}

\begin{figure}[h]
\centering 
\begin{tikzpicture}[scale=1]
\tkzDefPoints{0/0/A, 1/3/B, 5/0/C}
\tkzDrawPolygon(A,B,C)
\tkzDrawPoints[color=blue](A,B,C)
\tkzLabelPoints(A,C){A,C}
\tkzLabelPoints[above](B){B}
\tkzDrawBisector(A,B,C) \tkzGetPoint{D}
\tkzLabelPoints(D){D}
\tkzFillPolygon[color=yellow, opacity=.5 ](A,B,D)
\tkzFillPolygon[color=green, opacity=.5](D,B,C)
\tkzMarkAngle[size=0.7](A,B,D)
\tkzLabelAngle(A,B,D){$\alpha$}
\tkzMarkAngle[size=0.7](D,B,C)
\tkzLabelAngle(D,B,C){$\alpha$}

\tkzCentroid(A,B,D) \tkzGetPoint{M}
\tkzLabelPoint(M){$S_1$}
\tkzCentroid(D,B,C) \tkzGetPoint{N}
\tkzLabelPoint(N){$S_2$}

\tkzLabelSegment[below](A,D){m}
\tkzLabelSegment[below](D,C){n}
\end{tikzpicture}

\caption{}
   
\end{figure}
Si $\overline{BD}$ es bisectriz interior del triángulo $\triangle ABC$\\
Entonces: 
%formulas
$$ \frac{S_1}{S_2}=\frac{a}{b}=\frac{m}{n} $$

\newpage
%*********************
% EJERCICIO
\subsection{Teorema de Van Aubel}
\begin{figure}[h]
\centering 
\begin{tikzpicture}[scale=1]
\tkzDefPoints{0/0/A, 1/3/B, 5/0/C}
\tkzInCenter(A,B,C) \tkzGetPoint{J}
\tkzDrawPolygon(A,B,C)

\tkzDrawBisector(A,B,C) \tkzGetPoint{D}
\tkzDrawBisector(A,C,B) \tkzGetPoint{E}
\tkzDrawBisector(C,A,B) \tkzGetPoint{F}
\tkzInterLL(B,D)(C,E) \tkzGetPoint{K}
\tkzDrawPoint(K)
\tkzLabelPoint(K){$K$}

\tkzLabelSegment[left](B,E){m}
\tkzLabelSegment[left](E,A){n}
\tkzLabelSegment[left](B,K){x}
\tkzLabelSegment[left](K,D){y}
\tkzLabelSegment[right](B,F){p}
\tkzLabelSegment[right](F,C){q}
\end{tikzpicture}

\caption{}
   
\end{figure}
Sea $K$ el punto de intersección de las cevianas de un triángulo.\\
Se cumple: 
%formulas
$$ \frac{x}{y}=\frac{m}{n}=\frac{p}{q} $$

%*********************
% EJERCICIO
\subsection{Teorema de raíces de áreas triangulares}
\begin{figure}[h]
\centering 
\begin{tikzpicture}[scale=1]
\tkzDefPoints{0/0/A, 2/5/B, 5/0/C}
\tkzDrawPolygon(A,B,C)
\tkzDrawCircle[in](A,B,C) \tkzGetPoint{I}
\tkzDrawPoints(A,B,C)
\tkzDefPointBy[projection=onto A--C](I) \tkzGetPoint{M}
\tkzDrawPoint(M)
\tkzInterLC(A,I)(I,M) \tkzGetFirstPoint{E}
\tkzInterLC(B,I)(I,M) \tkzGetFirstPoint{R}
\tkzInterLC(C,I)(I,M) \tkzGetFirstPoint{T}

\tkzDefTangent[at=E](I) \tkzGetPoint{e}
\tkzInterLL(E,e)(A,B) \tkzGetPoint{J}
\tkzInterLL(E,e)(A,C) \tkzGetPoint{N}
\tkzDrawSegment(J,N)
\tkzDefTangent[at=R](I) \tkzGetPoint{r}
\tkzInterLL(R,r)(B,A) \tkzGetPoint{K}
\tkzInterLL(R,r)(B,C) \tkzGetPoint{M}
\tkzDrawSegment(K,M)
\tkzDefTangent[at=T](I) \tkzGetPoint{t}
\tkzInterLL(T,t)(C,B) \tkzGetPoint{P}
\tkzInterLL(T,t)(C,A) \tkzGetPoint{O}
\tkzDrawSegment(P,O)

\tkzFillPolygon[color=yellow, opacity=.5 ](A,J,N)
\tkzFillPolygon[color=green, opacity=.5](B,K,M)
\tkzFillPolygon[color=red, opacity=.5](C,P,O)

\tkzInCenter(A,J,N) \tkzGetPoint{Y}
\tkzLabelPoint(Y){$A_1$}
\tkzInCenter(B,K,M) \tkzGetPoint{U}
\tkzLabelPoint(U){$A_2$}
\tkzInCenter(C,P,O) \tkzGetPoint{H}
\tkzLabelPoint(H){$A_3$}


\end{tikzpicture}

\caption{}
   
\end{figure}
Si $A$ es el área de la región triangular $\triangle ABC$. Entonces: 
%formulas
$$ \sqrt{A}=\sqrt{A_1}+\sqrt{A_2}+\sqrt{A_3} $$

\newpage
%*********************
% EJERCICIO
\section{Propiedades de  triángulos}

\subsection{Propiedades de altura de ángulo doble en ángulo triple}
\begin{figure}[h]
\centering 
\begin{tikzpicture}[scale=2 ]
\tkzDefPoint(0,0){A}
\tkzDefPoint(5,0){B}
\tkzDefTriangle[two angles = 60 and 30](A,B) \tkzGetPoint{C}
\tkzDrawSegment(A,B)
\tkzDrawPoints(A,B)
\tkzLabelPoints(A,B)
\tkzDrawSegments(A,C B,C)
\tkzDrawPoints(C)
\tkzLabelPoints[above](C)
\tkzDefTriangle[two angles = 90 and 20](C,A) \tkzGetPoint{D}
\tkzDrawSegment(A,D)
\tkzDefPointBy[projection=onto A--D](C) \tkzGetPoint{F}
\tkzDefPointBy[projection=onto A--B](D) \tkzGetPoint{G}
\tkzDrawSegments(C,F D,G)
\tkzLabelSegment[below](C,F){k}
\tkzLabelSegment(D,G){x}
\tkzLabelAngle[pos=.6](D,A,C){$\theta$}
\tkzLabelAngle[pos=.6](B,A,D){$2\theta$}
\tkzMarkRightAngles[size=.15,fill=gray!15](C,F,D A,C,B D,G,B)

\end{tikzpicture}
\caption{}
\end{figure}
% ======
Sea el triángulo rectangulo $\angle ABC$. \\
Se cumple: 
%formulas
$$ x = 2k $$

