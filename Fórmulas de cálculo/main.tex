\documentclass{article}
\usepackage[utf8]{inputenc}
\usepackage{amsmath, amsthm, amssymb}
\usepackage{longtable}
\usepackage{physics}
\usepackage{amsfonts}

\usepackage[table]{xcolor}
\usepackage{mathtools}
\usepackage{esint}

\setlength{\arrayrulewidth}{0.5mm}
\setlength{\tabcolsep}{18pt}
\renewcommand{\arraystretch}{2.5}
\usepackage{graphicx}

\usepackage[a4paper]{geometry}
\geometry{top=1.5cm, bottom=1.0cm, left=1cm, right=1.25cm}

%\usepackage[table,xcdraw]{xcolor}
\usepackage{multirow}
\usepackage{array}

\newcommand{\dst}{\displaystyle}
\DeclareMathOperator{\arccosh}{arccosh}

% Spanish-specific commands
\usepackage[spanish]{babel}

\title{Cálculo}
\author{Javier ArGo}
\begin{document}
\tableofcontents
\maketitle
%\input{cuerpo/1_principal/principal}

\section{Fórmulas de cálculo}
\subsection{Derivadas}

%%%%%%%%%%%%%%%%%%%%%%%%%%%%%%%%%%%%
\begin{table}[!ht]
    \centering
    \begin{tabular}{|l|l|l|}
    \hline
   $\dfrac {\mathrm{d} C}{\mathrm{d} x}=0$ & $\dfrac {\mathrm{d} x}{\mathrm{d} x}=1$ &   $\dfrac {\mathrm{d} \left ( x^n \right )}{\mathrm{d} x}=nx^{n-1}$  \\ \hline  
   $\dfrac {\mathrm{d} \left ( ln (x) \right )}{\mathrm{d} x}=x^{-1}$ & $\dfrac {\mathrm{d} \left ( a^x \right )}{\mathrm{d} x}=a^x ln (a)$ & $\dfrac {\mathrm{d} \left ( e^x \right )}{\mathrm{d} x}=e^x$\\ \hline  
   $\dfrac {\mathrm{d} \left ( \sin ax \right )}{\mathrm{d} x}=a\cos ax$ & $\dfrac {\mathrm{d} \left ( \cos ax \right )}{\mathrm{d} x}=-a\sin ax$ & $\dfrac {\mathrm{d} \left ( \tan ax \right )}{\mathrm{d} x}=\dfrac a{\cos^2 ax}=a\sec^2ax$ \\ \hline
    $\dfrac {\mathrm{d} \left ( \cot x \right )}{\mathrm{d} x}=-\dfrac 1{\sin^2 x}=-\csc^2x$ & $\dfrac {\mathrm{d} \left ( \sec x \right )}
{\mathrm{d} x}=\sec x \tan x$ & $\dfrac {\mathrm{d} \left ( \csc x \right )}
{\mathrm{d} x}=-\csc x \cot x$  \\ \hline
$\dfrac {\mathrm{d} \left ( \arcsin x \right )}{\mathrm{d} x}=\dfrac 1{\sqrt{1-x^2}}$ & $\dfrac {\mathrm{d} \left ( \arccos x \right )}{\mathrm{d} x}=\dfrac {-1}{\sqrt{1-x^2}}$ & $\dfrac {\mathrm{d} \left ( \arctan x \right )}{\mathrm{d} x}=\dfrac 1{1+x^2}$ \\ \hline
$ \dfrac {\mathrm{d} \left ( \arctan x \right )}{\mathrm{d} x}=\dfrac 1{1+x^2}$ & $ \dfrac {\mathrm{d} \left ( arccot\:x \right )}{\mathrm{d} x}=\dfrac {-1}{1+x^2}$ & $ \dfrac {\mathrm{d} \left ( arcsec\:x \right )}{\mathrm{d} x}=\dfrac 1{x\sqrt{x^2-1}}$ \\ \hline
$ \dfrac {\mathrm{d} \left ( arccsc\:x \right )}{\mathrm{d} x}=\dfrac {-1}{x\sqrt{x^2-1}}$ & $ \dfrac {\mathrm{d} \left ( \sinh x \right )}{\mathrm{d} x}=\cosh x$ & $ \dfrac {\mathrm{d} \left ( \cosh x \right )}{\mathrm{d} x}=\sinh x$ \\ \hline
$ \dfrac {\mathrm{d} \left ( \tanh x \right )}{\mathrm{d} x}=\dfrac 1{\cosh^2 x}=sech^2x$ & $ \dfrac {\mathrm{d} \left ( \coth x \right )}{\mathrm{d} x}=-\dfrac 1{\sinh^2 x}=csch^2x$ & $ \dfrac {\mathrm{d} \left ( sech x \right )}{\mathrm{d} x}=-sech\:x \tanh x$ \\ \hline
$ \dfrac {\mathrm{d} \left ( csch x \right )}{\mathrm{d} x}=-csch\:x \coth x$ & $ \dfrac {\mathrm{d} \left ( arcsinh\:x \right )}{\mathrm{d} x}=\dfrac 1{\sqrt{1+x^2}}$ & $ \dfrac {\mathrm{d} \left ( arccosh\:x \right )}{\mathrm{d} x}=\dfrac {-1}{\pm \sqrt{x^2-1}}$ \\ \hline
$ \dfrac {\mathrm{d} \left ( arctanh\:x \right )}{\mathrm{d} x}=\dfrac 1{1-x^2}$ & 
 - & - \\
 \hline
  \end{tabular}
    \caption{Tabla de derivadas}
    \label{tab:Derivadas}
\end{table}









\newpage
\subsubsection{Propiedades de derivadas}
\begin{table}[h]
    \centering
    \begin{tabular}{|c|c|}
    \hline
     Multiplicacion por escalar    &  $\dfrac {\mathrm{d} (Cy)}{\mathrm{d} x}=C \dfrac {\mathrm{d} y}{\mathrm{d} x}$   \\ \hline
    Suma y distribución     & $\dfrac {\mathrm{d} (y+z)}{\mathrm{d} x}=\dfrac {\mathrm{d} y}{\mathrm{d} x}+\dfrac {\mathrm{d} z}{\mathrm{d} x}$\\ \hline
    Derivada del producto  &   $\dfrac {\mathrm{d} (y \cdot z)}{\mathrm{d} x}=z\dfrac {\mathrm{d} y}{\mathrm{d} x}+y\dfrac {\mathrm{d} z}{\mathrm{d} x}$ \\ \hline
    Derivada de la división  & $\dfrac {\mathrm{d} (\dfrac yz)}{\mathrm{d} x}=\dfrac {z\dfrac {\mathrm{d} y}{\mathrm{d} x}-y\dfrac {\mathrm{d} z}{\mathrm{d} x}}{z^2}$\\ \hline
    Regla de la cadena & $\dfrac {\mathrm{d} (f(g(x)))}{\mathrm{d} x}=f'(g(x))\dfrac {\mathrm{d} g(x)}{\mathrm{d} x}$\\ \hline
    \end{tabular}
    \caption{Propiedades}
\end{table}
%\newpage
\subsection{Integrales}
\footnotesize
%\resizebox{16.5cm}{!}{
\begin{longtable}{|l|l|}%[h]
%\centering


%\begin{tabular}
%\caption{A sample long table.} \label{tab:long} \\

\endfirsthead


\endhead


\hline
$ \dst \int a \:\dd x= ax+C $ & $ \dst \int a f(x) \:\dd x= a \int f(x)\:\dd x $ \\ \hline
$ \dst \int (f + g) \:\dd x= \int f \:\dd x+\int g \:\dd x $ &
$ \dst \int f \dd g= fg- \int g \dd f $ \\ \hline
\ensuremath{   
\begin{array}{c}
  \dst \int x^n \:\dd x=\dfrac { x^{n+1}}{n+1} +C \\
  n\neq1 
\end{array}  
}
& $ \dst \int x^{-1} \:\dd x=\ln|x|+C $ \\ \hline
$\dst \int e^{x} \:\dd x=e^{x}+C $ & $\dst \int a^{x} \:\dd x=\dfrac {a^{x}}{ln a}+C$ \\ \hline
$\dst \int xa^{x} \:\dd x=\dfrac {a^{x}}{ln a}\cdot \left ( x-\dfrac 1{\ln a}\right ) +C $ & $\dst \int xe^{x} \:\dd x=e^{x}\cdot\left ( x- 1 \right ) +C$ \\ \hline
$\dst \int \ln x \:\dd x=x\cdot \ln x - x +C=x\cdot (\ln x - 1) +C $ & $\dst \int x \ln x \:\dd x=\dfrac {x^2}4 \cdot (2 \ln x - 1) +C$ \\ \hline
$\dst \int \sin x \:\dd x=-\cos x +C $ & $\dst \int \cos x \:\dd x=\sin x +C$ \\ \hline
$\dst \int \sec x \tan x \:\dd x=\sec x +C $ & $\dst \int \csc x \cot x \:\dd x=-\csc x +C $ \\ \hline
$\dst \int \tan x \:\dd x=-\ln |\cos x| +C=\ln |\sec x| +C $ & $\dst \int \cot x \:\dd x=\ln |\cos x| +C=-\ln |\csc x| +C$ \\ \hline
$\dst \int \sec x \:\dd x=-\ln |\sec x+ \tan x| +C $ & $\dst \int \csc x \:\dd x=-\ln|\csc x- \cot x| +C$ \\ \hline
$\dst \int \sin^2 x \:\dd x=\dfrac x2 -\dfrac 14\sin 2x +C $ & $\dst \int \cos^2 x \:\dd x=\dfrac x2 +\dfrac 14\sin 2x +C$ \\ \hline
$\dst \int \tan^2 x \:\dd x=\tan x- x +C $ & $\dst \int \cot^2 x \:\dd x= -\cot x- x +C $ \\ \hline
$\dst \int \sec^2 x \:\dd x=\tan x +C$ & $\dst \int \csc^2 x \:\dd x=-\cot x +C$ \\ \hline
$\dst \int x\sin x \:\dd x=\sin x-x\cos x +C$ & $\dst \int x\cos x \:\dd x=\cos x+x\sin x +C$ \\ \hline
$\dst \int \arcsin x \:\dd x=x\sin x+\sqrt{1-x^2} +C$ & $\dst \int \arccos x \:\dd x=x\cos x-\sqrt{1-x^2} +C$ \\ \hline
$ \dst \int \arctan x \:\dd x=x\tan x-\ln(\sqrt{1+x^2}) +C $ & $\dst \int arccot\:x \:\dd x=x\cot x+\ln(\sqrt{1+x^2}) +C $ \\ \hline



\centering
\ensuremath{   
\begin{array}{c}
\dst \int \arcsec x \:\dd x=x\sec x-\ln(x+\sqrt{x^2-1}) +C  \\
=x\sec x-\arccosh x +C 
\end{array}  
} 
& \ensuremath{   \begin{array}{c} \dst \int \arccsc x \:\dd x=x\csc x+\ln(x+\sqrt{x^2-1}) +C  \\ =x\sec x+\arccosh x +C     \end{array}   } 
\\ \hline
$\dst \int \sinh x \:\dd x=\cosh x +C$ & $\dst \int \cosh x \:\dd x=\sinh x +C $ \\ \hline
$\dst \int sech^2 x \:\dd x=\tanh x +C$ & $\dst \int csch^2 x \:\dd x=-\coth x +C$ \\ \hline
$\dst \int sech x \tanh x\:\dd x=-sech\:x +C$ & $\dst \int csch x \coth x\:\dd x=-csch\:x +C$ \\ \hline
$\dst \int \tanh x\:\dd x=\ln (\cosh x )+C$ & $\dst \int \coth x\:\dd x=\ln |\sinh x |+C$ \\ \hline
$\dst \int sech\:x \:\dd x=\arctan (\sinh x )+C$ & $\dst \int csch\:x\:\dd x=arccoth(\cosh x )+C=\ln \tanh (\frac x2 )+C $ \\ \hline
$ \dst \int \dfrac 1{x^2+a^2}\:\dd x=\dfrac 1a\arctan \dfrac xa+C=-\dfrac 1a arccot \dfrac xa+C$ &
\ensuremath{   
\begin{array}{c}
 \dst \int \dfrac 1{x^2-a^2}\:\dd x=\dfrac 1{2a}\ln(\dfrac {x-a}{x+a})+C \\ 
 x^2>a^2 
\end{array}  
}
\\ \hline
\ensuremath{   
\begin{array}{c}
 \dst \int \dfrac 1{a^2-x^2}\:\dd x=\dfrac 1{2a}\ln(\dfrac {a+x}{a-x})+C\\x^2<a^2
\end{array}  
}
 & $\dst \int \dfrac 1{\sqrt{a^2-x^2}}\:\dd x=\sin \dfrac xa+C=-\cos \dfrac xa+C$ \\ \hline
$\dst \int \dfrac 1{\sqrt{x^2\pm a^2}}\:\dd x=\ln( x+\sqrt{x^2\pm a^2})+C$ & $\dst \int \dfrac 1{x\sqrt{a^2\pm x^2}}\:\dd x=\frac 1a \ln | \dfrac {x}{a+\sqrt{a^2\pm x^2}}|$ \\ \hline
$\dst \int \dfrac 1{x\sqrt{x^2-a^2}}\:\dd x=\frac 1a\arccos \dfrac {a}{x}=-\frac 1a arcsec \dfrac {x}{a}+C$ & $\dst \int \sqrt{a^2-x^2}\:\dd x=\frac x2 \sqrt{a^2-x^2}+\dfrac {a^2}{2} \arcsin \dfrac {x}{a}+C$ \\ \hline
$\dst \int \sqrt{x^2\pm a^2}\:\dd x=\frac x2 \sqrt{x^2\pm a^2}\pm \dfrac {a^2}{2} \ln (x+\sqrt{x^2\pm a^2})+C $ & $\dst \int e^{ax} \sin bx \:\dd x=\dfrac {e^{ax}(a \sin bx- b\cos bx)}{a^2+b^2}+C $ \\ \hline
$\dst \int e^{ax} \cos bx \:\dd x=\dfrac {e^{ax}(a\cos bx+ b \sin bx)}{a^2+b^2}+C $ & - \\ \hline

%\end{tabular}


%\caption{Integrales}
%\label{tab:integrales}
\end{longtable}
%}

%\newpage
\subsubsection{Propiedades de Integrales Definidas}
\begin{table}[h]
    \centering
    \begin{tabular}{|c|}
    \hline
$\dst \int_a^b (f(x)\pm g(x)) \dd x=\int_a^b f(x)\dd x \pm \int_a^b g(x) \dd x$ \\ \hline
$\dst \int_a^b C \cdot f(x) \dd x=C \cdot \int_a^b f(x) \dd x$ \\ \hline
$\dst \int_a^b f(x) \dd x=\int_a^c f(x) \dd x + \int_c^b f(x) \dd x$ \\ \hline
$\dst \int_a^b f(x) \dd x=-\int_b^a f(x) \dd x$ \\ \hline
$\dst \int_a^a f(x) \dd x=0$ \\ \hline
    \end{tabular}
    \caption{Propiedades de integrales definidas}
    \label{tab:prop-integral-def}
\end{table}
\newpage
\subsection{Transformadas de Laplace}
\begin{table}[h]
    \centering
   \resizebox{7.2cm}{!} {
\begin{tabular}{|c|c|}
\hline                          
$F(s)	$&$	f(t) $ \\ \hline
$\dfrac 1{s}	$ &	 $1$ \\ \hline
$\dfrac 1{s^2}	$& $	t$ \\ \hline
$\dfrac 1{s^n}	$& $	\dfrac {t^{n-1}}{(n-1)!}$ \\ \hline
$\dfrac 1{s \pm a}	$& $	e^{\mp at}$ \\ \hline
$\dfrac 1{s(s + a)}	$& $	\dfrac 1a(1-e^{-at})$ \\ \hline
$\dfrac 1{s^2(s + a)}	$& $	\dfrac 1{a^2}(e^{-at}+mt-1)$ \\ \hline
$\dfrac a{s^2 + a^2}	$& $	\sin at$ \\ \hline
$\dfrac s{s^2 + a^2}	$& $	\cos at$ \\ \hline
$\dfrac a{s^2 - a^2}	$& $	\sinh at$ \\ \hline
$\dfrac s{s^2 - a^2}	$& $	\cosh at$ \\ \hline
$\dfrac 1{s(s^2 + a^2)}	$& $	\dfrac 1{a^2}(1-\cos at)$ \\ \hline
$\dfrac 1{s^2(s^2 + a^2)}	$& $	\dfrac 1{a^3}(at-\sin at)$ \\ \hline
$\dfrac 1{(s+a) \cdot (s+b)}	$& $	\dfrac 1{a-b}(e^{-bt}-e^{-at})$ \\ \hline
$\dfrac s{(s+a) \cdot (s+b)}$&	 $	\dfrac 1{b-a}(be^{-bt}-ae^{-at})$ \\ \hline
$\dfrac 1{(s + a)^2}	$& $	te^{- at}$ \\ \hline
$\dfrac 1{(s + a)^n}	$& $	\dfrac {t^{n-1}}{(n-1)!}e^{- at}$ \\ \hline
$\dfrac s{(s + a)^2}	$& $	e^{- at}(1-at)$ \\ \hline
$\dfrac 1{(s^2 + a^2)^2}	$& $	\dfrac 1{2a^3}(\sin at -at \cos at)$ \\ \hline
$\dfrac s{(s^2 + a^2)^2}	$& $	\dfrac t{2a}(\sin at )$ \\ \hline
$\dfrac {s^2}{(s^2 + a^2)^2}	$& $	\dfrac 1{2a}(\sin at +at \cos at)$ \\ \hline
$\dfrac {s^2-a^2}{(s^2 + a^2)^2}	$& $	t \cos at$ \\ \hline
                          
\end{tabular}
}

    \caption{Transformadas de Laplace}
    \label{tab:transfor-laplace}
\end{table}
\newpage
\subsubsection{Propiedades de transformadas de Laplace}
\begin{table}[h]
\centering
\resizebox{16.5cm}{!}{
\begin{tabular}{|c|c|}
\hline
\multicolumn{1}{|c|}{Propiedad}     & Fórmula \\ \hline
Definición	& $	\dst F(s)=\mathcal {L} \{ f(t) \}= \int_0^{\infty} e^{-st}f(t) \dd t $  \\ \hline
Linealidad	& $	\mathcal {L} \{ a \cdot f(t) +b \cdot g(t)\}= a \cdot \mathcal {L} \{f(t) \}+b \cdot \mathcal {L} \{ g(t)\}$ \\ \hline
Derivación 
& \ensuremath{   
\begin{array}{c}
  \mathcal {L} \{ f'(t) \}=s \mathcal {L} \{ f(t) \}-f(0)\\ 
  \mathcal {L} \{ f''(t) \}=s^2 \\ 
  \mathcal {L} \{ f(t) \}-s f(0)-f'(0) \\ 
  \mathcal {L} \{ f^{(n)}(t) \}= s^n \mathcal {L} \{ f(t) \}-s^{n-1} f(0)-\ldots-f^{(n-1)}(0) 
  \\ \mathcal {L} \{ f^{(n)}(t) \}= s^n \mathcal {L} \{ f(t) \}-\dst \sum_{i=1}^n s^{n-i}f^{(i-1)}(0)
\end{array}  
} \\ \hline
Integración	& $	\mathcal {L} \{ \int_{0-}^t f(\tau) \dd \tau \}=\dfrac 1s \mathcal {L} \{ f(t) \}$ \\ \hline
Dualidad& $		\mathcal {L} \{ t \cdot f(t) \}=-F'(s)$ \\ \hline
Desplazamiento de frecuencia	& $	\mathcal {L} \{ e^{at}f(t) \}=F(s-a)$ \\ \hline
Desplazamiento temporal	
& \ensuremath{   
\begin{array}{c}
  \mathcal {L} \{ f(t-a) \cdot u(t-a) \}=e^{-as} F(s) \\ \mathcal {L}^{-1} \{ e^{-as} F(s) \}=f(t-a) \cdot u(t-a) \\ u(t-a)=\left\{ \begin{array}{cl}0 & \text{si}\ t < a \\1 & \text{si}\ t \ge a \end{array}\right.
\end{array}  
} \\ \hline

Convolución	& $	\mathcal {L} \{ f(t) \cdot g(t) \}=F(s) \cdot G(s)   $ \\ \hline

\end{tabular}}
\caption{Propiedades de Transformadas de Laplace}
\label{tab:prop-transf-laplace}
\end{table}

\end{document}
